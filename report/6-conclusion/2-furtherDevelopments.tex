\chapter{Possible Further Developments}
There are many features that were left out of the application that can possibly be implemented for a next increment of the application:

\begin{itemize}
  \item Remind users to enter their expenses through notifications
  \item Tell users how much they have saved through notifications
  \item Set an initial automatic budget based on income
  \item Allow the modification of entered expenses
  \item Alert users when daily budget is about to be reached
  \item Allow separation of expenses that count towards daily expenses or not
  \item Allow users to enter their income in the same way they get paid
  \item Allow the users to search for specific expenses by date or throughout a range of dates
  \item Calculate the daily spend allowance based on the number of days of the month
  \item Allow users to change the currency used in the application
\end{itemize}

Besides the implementation of these features, there is also a list of things that could be done to improve the application without adding new features:
\begin{itemize}
  \item Re-design the application to give it a more modern look
  \item Implement the application in Android Studio
  \item Improve the data storage solution and structure
  \item Implement the dynamic population of screens with UI elements
  \item Implement data input verification
\end{itemize}

Due to time constraints, there were also some errors and small details that were left in the application. Those would also need fixing if the application were to have an increment.
