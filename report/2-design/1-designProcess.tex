\chapter{Design Process}
After planning what features I would implement on the app and planning the tasks that I would do in order to get the app ready on time for the presentation, I tackled the design of the application by starting to design the UI of the app in paper, then moving to wireframes and review them, reworking both if necessary and then implement the UI in App Inventor.

Since I didn't know much about how App Inventor worked and how to build applications with it when I started designing the application, I decided to dive in into research of how to make applications with App Inventor and starting implementing the features as I learned about the platform and then rework on the final design taking into account the constraints of building in App Inventor.

When designing the User Interface, I assumed that the application would be used with the smartphone screen being in a portrait orientation and took into consideration that the elements would need enough screen space to allow the user to interact with them. This task was made easy for me considering that App Inventor automatically allocates a generous amount of space for the elements as a default, but it was still something that I had to take into account when modifying the original proportions of the elements in order to achieve a certain design.

Considering that most of the functionality of the application would be triggered by events resulting of the interaction of the user with the User Interface, I started planning the design of the application by deciding on what the User Interface would be.
