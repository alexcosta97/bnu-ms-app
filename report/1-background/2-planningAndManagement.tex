\chapter{Project Planning and Management}
In order to manage my progress in this project, I am using some Scrum concepts applied to the way GitHub manages project management, since it is where I am hosting the code for the report.

GitHub uses Kanban boards to help teams keep track of the tasks that need to be completed, in a section of the repository called Projects. I used an automated Kanban board that would move my cards around as I got around completing the tasks.

Tasks can be defined in GitHub through the creation of issues. Issues can contain anything from feature implementation and basic project tasks to bugs that need fixing. I would use a issue to note a big task, for example, design the UI of the app, and use the checkboxes available to the writing of an issue to add smaller subtasks that can be considered as objectives to get the bigger task done, such as, for example, draw the sketches of the UI.

I also used the concept of Sprints from the Scrum methodology to set myself a deadline for a certain number of tasks that I needed to complete that were related (for example Design for all the tasks related to the UI of the app, from sketches and wireframes to the actual implementation of the app UI). In GitHub, we can implement Sprints by setting milestones with a title, description and date for completion and then assigning the tasks to the milestone that we want it to be related to.

For my project, I set myself the following milestones and tasks. The plan will be organized in the following way:
\begin{itemize}
  \item Milestone
  \begin{itemize}
    \item Task
    \begin{itemize}
      \item Subtask
    \end{itemize}
  \end{itemize}
\end{itemize}

\section{Sprints and tasks}
\begin{itemize}
  \item Design and Planning
  \begin{itemize}
    \item Decide on the features to be implemented
    \begin{itemize}
      \item Brainstorm features for the app
      \item Select most basic features
      \item Narrow down list of optional features to the three most important
    \end{itemize}
    \item Plan implementation of the app
    \begin{itemize}
      \item Write down all the tasks that need to be done for the completion of the app
      \item Plan sprints
    \end{itemize}
    \item Design the app
    \begin{itemize}
      \item Design first sketches
      \item Implement sketches in wireframe and review
      \item Rework on the sketches and re-implement if and as necessary
    \end{itemize}
  \end{itemize}
  \item Implementation
  \begin{itemize}
    \item Prepare UI and Layout of the app (for the following features)
    \begin{itemize}
      \item Management and customization of budget
      \item Input of expenses
      \item Input of income
      \item Tell users how much they have been saving
      \item Separate expenses into different categories
      \item Track daily expenses
    \end{itemize}
    \item Implementation of features related to the UI (Implement processes that happen when the user interacts with the UI for the following features)
    \begin{itemize}
      \item Management and customization of budget
      \item Allow input of expenses
      \item Allow input of income
      \item Tell users how much they have been saving
      \item Separate expenses into different categories
      \item Track daily expenses
    \end{itemize}
    \item Implement background features
    \begin{itemize}
      \item Reset budget and money when payday has been reached
      \item Update total savings when payday comes through
      \item Track daily expenses (reset money spent for the day)
    \end{itemize}
  \end{itemize}
  \item Deployment and testing
  \begin{itemize}
    \item Deploy the app to a local emulator
    \begin{itemize}
      \item Deploy the app into the emulator provided with App Inventor
    \end{itemize}
    \item Deploy the app on your own mobile device
    \begin{itemize}
      \item Make sure that your device is available
      \item Deploy the app into the device
    \end{itemize}
    \item Test the app in the local emulator (things to test)
    \begin{itemize}
      \item Interaction with the design (the application does what it should when the user interacts with it)
      \item All the data is being stored and retrieved
      \item All the calculations are correct
    \end{itemize}
    \item Test the app on your own mobile device
    \begin{itemize}
      \item Interaction with the design (application does what it should when the user interacts with it)
      \item All the data is stored and retrieved
      \item All the calculations are correct
      \item Make sure it doesn't crash
    \end{itemize}
  \end{itemize}
\end{itemize}

Another good thing about planning the progress of the project with GitHub is that it allows me to review and modify the issues (or tasks) whenever I want, so if there is new things to implement or things that need to be done I can quickly review my project plan, but it also allows me to post comments related to the issue that I am working on, allowing me to track and log my progress, difficulties and sucesses when doing my work.
