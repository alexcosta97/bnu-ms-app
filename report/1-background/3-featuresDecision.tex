\chapter{Deciding on Features}
When I tackled the task of deciding on which features to implement, I started by concentrating for about five minutes and write down all the features that I thought I would like to see on the app.

When I first started thinking about this project, I also wanted the app to be able to remind the users to input their expenses and to encourage users to keep on tracking their expenses by using notifications to congratulate them on how much they have been saving so far, but, unfortunately, due to time constraints, I had to give up on that idea. However, that idea also inspired me to get some more original ideas of features for the application and it probably wouldn't be the same as what it is now if I didn't have the idea of trying to encourage the user to use the app as much as possible.

These are the features that I came up with on my brainstorm session:

\begin{itemize}
  \item Set automatic budget based on income
  \item Allow management and customization of budget
  \item Separate expenses into different categories
  \item Allow input of expenses
  \item Allow modification and removal of entered expenses (categories and retification of amounts)
  \item Calculate daily expenses based on budget
  \item Track daily expenses
  \item Remind users to enter their expenses (at opportunate times)
  \item Tell users how much they have been saving
  \item Alert users when daily budget is about to be reached
  \item Allow separation of expenses that count towards daily expenses or not
  \item Allow users to enter their income in a way that accomodates the different ways of getting paid (per hour or annual salary) and when they get paid (weekly, fortnightly or monthly)
\end{itemize}

After the brainstorm session, I separated all the features that I thought about into two types: the ones that would provide the basic functionality of the app and the ones that were optional for the implementation of a budget planner. When I first did this separation, I was still planning on getting the application to motivate the user through notifications.

\section{Basic Features}
\begin{enumerate}
  \item Management and customization of budget
  \item Allow input of expenses
  \item Allow input of income
  \item Remind users to enter their expenses
  \item Tell users how much they have been saving
\end{enumerate}

\section{Optional Features}
\begin{enumerate}
  \item Set automatic budget based on income
  \item Separate expenses into different categories
  \item Allow modification and removal of entered expenses
  \item Calculate daily expenses based on budget
  \item Track daily expenses
  \item Alert users when daily budget is about to be reached
  \item Allow separation of expenses that count towards daily expenses or not
  \item Allow users to enter their income in a way that accomodates the different ways of getting paid (per hour or annual salary based) and when they get paid (weekly, fortnightly or monthly)
\end{enumerate}

When I separated all these expenses, I had to reduce my list of optional features to implement. For me it was pretty straight forward that all the basic features had to be implemented and that only a couple of the optional features would end up being implemented. At the time of writing this section of the report, I am now realising that I ended up implementing more optional features than what I had initially intended because of the way I designed the app and because it seemed like the application wouldn't be complete without them.

\section{Final features}
\begin{enumerate}
  \item Management and customization of budget
  \item Allow input of expenses
  \item Allow input of income
  \item Separate expenses into different categories
  \item Allow removal of entered expenses
  \item Calculate daily expenses based on budget
  \item Track daily expenses
  \item Tell users how much they have been saving
\end{enumerate}
